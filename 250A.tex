\documentclass[12pt]{book}
 
\usepackage[margin=1in]{geometry} 
\usepackage{amsmath,amsthm,amssymb, amscd}


\usepackage{amsfonts}

\usepackage{ stmaryrd }

\usepackage{ amscd}
 
\newcommand{\N}{\mathbb{N}}
\newcommand{\Z}{\mathbb{Z}}
 
\newenvironment{theorem}[2][Theorem]{\begin{trivlist}
\item[\hskip \labelsep {\bfseries #1}\hskip \labelsep {\bfseries #2.}]}{\end{trivlist}}
\newenvironment{lemma}[2][Lemma]{\begin{trivlist}
\item[\hskip \labelsep {\bfseries #1}\hskip \labelsep {\bfseries #2.}]}{\end{trivlist}}
\newenvironment{exercise}[2][Exercise]{\begin{trivlist}
\item[\hskip \labelsep {\bfseries #1}\hskip \labelsep {\bfseries #2.}]}{\end{trivlist}}
\newenvironment{problem}[2][Problem]{\begin{trivlist}
\item[\hskip \labelsep {\bfseries #1}\hskip \labelsep {\bfseries #2.}]}{\end{trivlist}}
\newenvironment{question}[2][Question]{\begin{trivlist}
\item[\hskip \labelsep {\bfseries #1}\hskip \labelsep {\bfseries #2.}]}{\end{trivlist}}
\newenvironment{corollary}[2][Corollary]{\begin{trivlist}
\item[\hskip \labelsep {\bfseries #1}\hskip \labelsep {\bfseries #2.}]}{\end{trivlist}}
 
\usepackage[utf8]{inputenc}
 \begin{document}
 
 \chapter{Group Theory}
 
  \section{Basics}
 
 \section{Group Action}
 
 \section{Sylow Theorems}
 
 \section{Free Abelian Groups}
 
\chapter{Commutative Algebra}
 \section{Rings}
 Rings are groups with multiplication defined. By definition the group (which is usually additive) is abelian. An ideal of a ring is the equivalent of a normal group. It is the kernel of some homomorphism. It is clearly closed under operation by its own elements, but the crux of ideals is that they are closed under multiplication by elements not in the ideal. There are different types of ideals:
 \begin{itemize}
 \item A \textbf{prime} ideal is an ideal such that if $ab=\mathfrak{p}$ then $a \in \mathfrak{p}$ or $b \in \mathfrak{p}$.
 \item A \textbf{maximal} ideal is an ideal s.t if $M \subset I$ then $I=R$.
 \item A \textbf{principal} ideal is an ideal s.t each $I=(a)$ for some $a \in R$.
 \end{itemize}
 Notice that every maximal ideal is prime but the converse is not generally true (but in Dedekind domains it is). 
 
 \subsection{Noetherian Rings}
 Noetherian rings are rings with extra structure. There are three equivalent conditions for a ring to be Noetherian:
 \begin{enumerate}
 \item Ascending Chain Condtion (ACC): A chain of ideals is finite. This means if you have $I_1 \subset \ldots \subset I_n \subset \ldots$ then for large enough $n$ we have $I_{n}=I_{n+1}$.
 \item Every ideal is finitely generated. There is a finite set of elements s.t that every element in the ring is a linear combination of these elements.
 \item Every nonempty set of ideals partially ordered by inclusion has a maximal element.
 \end{enumerate}
  \subsection{Local Rings}
 We say a ring is \textbf{local} if it has exactly one maximal ideal. Suppose $\mathfrak{p}$ is a maximal ideal in $R$ and let $S=R \setminus \mathfrak{p}$. Then $S^{-1} R$ is the local ring at $\mathfrak{p}$ 
 \subsection{Dedekind Domains}
 
 \section{Modules}
 We expand the notion of an abelian group with adding on action of a ring on the group elements. An $A$-module is an abelian group with a ring homomorphism $A \rightarrow \text{End}(M)$. We have already seen modules:
 \begin{itemize}
 \item When $A$ is a field the module is a vector space.
 \item An ideal $\mathfrak{a}$ in $A$ is an $A$-module.
 \item An abelian group is a $\mathbb{Z}$-module.
 \end{itemize}
 \section{Tensor Products}
 First we define a \textbf{bilinear} map, which is a map $M \times N$ s.t $(m,n)$ is linear when $m$ is fixed and linear where $n$ is fixed. We have the free module $A^{M \times N}$ and quotient it by the group generated by 
 \begin{align*}
 &(m+m',n)-(m,n)-(m',n) \\
 &(m,n+n')-(m,n)-(m,n')\\
 &(am,n)-a(m,n)\\
 &(m,an)-a(m,n)
 \end{align*}
 which we denote as $M \otimes N$. We have it that every bilinear map $M \times N \rightarrow P$ maps through $M \otimes N$, that is
 %diagram%
 This means for every bilinear map $M \times N \rightarrow P$ there exists a linear map $M \otimes N \rightarrow P$. Usually when writing a proof for tensor products you first construct the bilinear map and then use that fact.
 
 There are several identities for tensor products:
 \begin{itemize}
 \item $M \otimes N \cong M \otimes N$
 \item $M \otimes (N \otimes P)\cong(M \otimes N) \otimes P \cong M \otimes N \otimes P$
 \item $M \otimes A \cong M$
 \end{itemize}
 
Tensor products can also be described categorically.
 
 
 \section{Flatness}
 We now look at the properties of tensor products in exact sequences.
 \chapter{Polynomials}
 
\chapter{Field Theory}
 
\section{Algebraic Extensions}
Field theory is focused on making fields larger. If $F \subset E$ we call $E$ an \textbf{extension} of $E$. If you think of $E$ as just an abelian additive group with multiplication by $F$ then $E$ is a vector space over $F$. The dimension of the vector space is the \textbf{degree} of $E$ over $F$. We want to make fields larger by adding elements to the field. However once you add an element to a field, you also get all of its linear combinations of its powers with multiplication by the elements of the vector space. That is if we add $\alpha$ to $F$ then we also get $a_0+a_1 \alpha + a_2 \alpha^2 + \cdots$ where $a_i \in F$. Its natural to wonder if these linear combinations terminate. This is the idea of algebraic and transcendental numbers. An element $\alpha$ is \textbf{algebraic} over $F$ if it is a root of some polynomial $p(x) \in F[x]$. We denote $F[\alpha]$ as extending the field $F$ with the element $\alpha$. If $\alpha$ is algebraic then naturally we have $F[\alpha] \cong F[x]/(p(x))$. Here are the key results about algebraic extensions:
\begin{itemize}
  \item Every finite extension is an algebraic extension. Suppose $k \subset K$ with degree $n$ and pick some $\alpha \in K$. Then $\{ 1,\alpha, \alpha^2, \ldots, \alpha^{n} \}$ cannot be linearly independent which means not all coefficients are zero.
  \item Extensions are multiplicative. That is $[K:k]=[K:E][E:k]$ where $k \subset E \subset K$.
  \item The degree of an extension is the degree of the minimal polynomial. This is proven by showing $\{ 1,\alpha, \alpha^2, \ldots, \alpha^{d-1} \}$ is linearly independent.
  \item The minimal field containing $\alpha$ is the same as the field with $\alpha$ as its variable. That is $F(\alpha)=F[\alpha]$
\end{itemize}
 
 \section{Normal Extensions}
We know emphasis embedding fields into other fields. If we have $\sigma: k \rightarrow L$ and $k \subset K$ we say that $\tau: K \rightarrow L$ \textbf{extends} $\sigma$ when the restriction on $k$ is $\sigma$.

Before we discussed how extending fields amounts to adding a root of an irreducible polynomial. We now as the question in the other direction: given a function can we extend the field so that the polynomial has a root? Can we extend it  so that it has all of its roots? Can we create an extension where every polynomial has all of its roots? 

Its important to notice that field homomorphism permute roots of polynomials. If $f=a_n x^n + \ldots + a_0$ and $\alpha$ is a root we have
\begin{align*}
\sigma(a_n \alpha^n+ \ldots+a_0=\sigma(a_n)(\sigma(\alpha))^n+\ldots + \sigma(a_0)
\end{align*}
 So as you can see $\sigma(\alpha)$ is a root of this new polynomial. 
\section{Separable Extensions}
Suppose $k \subset K$ is finite. How many automorphisms are there of $K$ over $k$? This is the same question as asking if the irreducible polynomial that extends $K$ has multiple roots. This is because $K$ is generated by the roots it adjoins, and automorphisms of $K$ over $k$ permute the roots of polynomials. Thus if roots are repeated, you "get less automorphisms that you potentially can fully get".
  

\section{Finite Fields}
We start by looking at the characteristic of a finite field. We have the map $\mathbb{Z} \rightarrow K$ where $K$ is a field and $1 \mapsto 1$. Clearly the kernel is nonzero because $K$ is finite. The kernel then is an ideal $n \mathbb{Z}$. However we cannot map zero divisors into a field, so the kernel is $p \mathbb{Z}$ where $p$ is prime. Thus $\mathbb{F}_p \hookrightarrow K$. We say that $K$ has characteristic $p$.  We can now think of $K$ as a vector space over $\mathbb{F}_p$. Clearly the dimension of this vector space is finite, say $n$. Thus we have that every field has cardinality $q=p^n$. If we look at the multiplicative subgroup of $K$. We have that $x^{q-1}=1$ for every element. This means every element satisfies the polynomial
\begin{align*}
f(x)=x^{q}-x
\end{align*}
This polynomial has at most $q$ elements, which it does, so the polynomial is separable. Another way of proving $f$ is separable is by seeing that $f'(x)=-1$ so it cannot have a multiple root.

We now see that the roots of the polynomial are closed under addition, subtraction, multiplication and division so form the unique finite field. Thus every finite field is the splitting field (which is unique) of $f$. 

The next important thing to show are the automorphism of $K$ over $\mathbb{F}_p$ is the Frobenius automorphism:
\begin{align*}
K &\rightarrow K \\
x &\mapsto x^p
\end{align*} 

\chapter{Galois Theory}

\section{Main Theorem}

The main idea behind Galois theory is that if $k \subset K$ is a galois extension then there is a bijection between the subfields of $K$ containing $k$ and the subgroups of $G$.  Let $H < G$ and $k \subset E \subset K$. The correspondence is given by:
\begin{align*}
 H &\mapsto K^H \\
 \text{Aut}(K/E) &\mapsfrom E
\end{align*}
The main theory states that this is a bijection. Most people, including Ribet in lecture, prove this by showing both maps are inverses to one another, which basically means $\text{Aut}(K/K^H)=H$ and $K^{\text{Aut}](K/E)}=E$. However Lang actually does this by proving $E \mapsto G(K/E)$ is injective and surjective. 

Suppose $k \subset E \subset K$. The next part of the theorem states that $k \subset E$ is normal iff its Galois group $H$ is normal in $G$. 
\section{Applications}
\subsection{$\mathbb{C}$ is Algebraically Closed}

 \chapter{Category Theory}
 

\end{document}